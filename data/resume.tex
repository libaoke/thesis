\begin{resume}

  \resumeitem{个人简历}

  1989 年 5 月 21 日出生于 黑龙江 省 青冈 县。

  2007 年 9 月考入 哈尔滨工业 大学 软件 学院 软件工程 专业,2011 年 7 月本科毕业并获得 工学 学士学位。

  2011 年 9 月免试进入 清华 大学 计算机 系攻读 博士 学位至今。

  \researchitem{发表的学术论文} % 发表的和录用的合在一起

  % 1. 已经刊载的学术论文(本人是第一作者,或者导师为第一作者本人是第二作者)
  \begin{publications}
    \item \textbf{Weichao Guo}, Kang Chen, Huan Feng, Yongwei Wu, Rui Zhang, and 
    Weiming Zheng. MARS: Mobile Application Relaunching Speed-Up through 
    Flash-Aware Page Swapping. IEEE Transactions on Computers (TC), vol. 
    65, no. 3, pp. 916-928, 2016. (CCF 推荐 A 类期刊)
    \item \textbf{Weichao Guo}, Kang Chen, Yongwei Wu, and Weiming Zheng. Bidding 
    for Highly Available Services with Low Price in Spot Instance Market.
     In Proceedings of the 24th International ACM Symposium on High-Performance 
     Parallel and Distributed Computing (HPDC), pp. 191-202, 2015. (CCF 推荐 B 类会议)
    \item Yongwei Wu, \textbf{Weichao Guo}, Jinglei Ren, Xun Zhao and Weiming Zheng. 
    NO2: Speeding up Parallel Processing of Massive Compute-Intensive Tasks.
     in IEEE Transactions on Computers (TC), vol. 63, no. 10, pp. 2487-2499, 
     2014. (CCF 推荐 A 类期刊)
  \end{publications}

  % 2. 尚未刊载,但已经接到正式录用函的学术论文(本人为第一作者,或者
  %    导师为第一作者本人是第二作者)。
  \begin{publications}[before=\publicationskip,after=\publicationskip]
    \item \textbf{Weichao Guo}, Kang Chen, Yongwei Wu, and Weiming Zheng. Gemini: 
    Hosting Online Services on Spot Markets Using Light-Weight Warm Standby Pair.
    In submission.
  \end{publications}

  % 3. 其他学术论文。可列出除上述两种情况以外的其他学术论文,但必须是
  %    已经刊载或者收到正式录用函的论文。
  %\begin{publications}
  %\end{publications}

  \researchitem{专利成果} % 有就写,没有就删除
  \begin{achievements}
    \item 武永卫, 郑纬民, 陈康, \textbf{郭维超}. 计算密集型并行任务的异常检测方法及系统: 
    中国, CN103645961A. (中国专利公开号)
    \item 郑纬民, \textbf{郭维超}, 武永卫, 陈康. 保证在线联机服务可用性的方法及装置: 
    中国, CN105577825A. (中国专利公开号)
  \end{achievements}

  \researchitem{参与的科研项目}
  \begin{achievements}
    \item 国家973课题: 高通量计算系统的云计算服务环境, 2011年, 2011CB302505
  \end{achievements}

\end{resume}
