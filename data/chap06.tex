\chapter{总结与展望}
\label{cha:conclusion_futruework}
竞价型云计算平台的出现为云计算的发展带来了新的变化,计算资源作为互联网中的 ``水'' 和 ``电'' 通过市场供需关系定价是一个新的趋势。本文从云租户的角度,分析了这一新型的低成本计算资源提供形式在各类系统中可能存在的利用机会。通过对竞价型云计算平台中已有的相关优化工作的调研发现:利用竞价型实例执行可中断的计算任务方面已经有不少研究工作,有的通过预估使用各类容错机制根据可能带来的开销综合选择最优的容错策略,有的通过对计算框架进行有针对性的修改实现了流式的任务结果输出从而避免了频繁做检查点的运行时开销,有的针对有截止期限等 SLA 的任务考虑竞价型实例的价格历史数据分布,试图给出保证任务满足 SLA 的竞价策略。也有研究者指出对于大规模并行计算任务,简单的将竞价型实例加入其中不一定能加速作业完成反而可能拖慢作业的执行。对于不可中断的服务如何利用竞价型实例的研究则相对缺乏。

本文的工作从这些方面入手,从单纯利用竞价型实例作为低成本计算资源,到在系统设计中融入对竞价型实例的支持,研究了竞价型云计算平台中如何提升系统可用性及性能的方法:1)对于大规模计算密集型并行任务处理,主要研究了如何利用竞价型实例解决普遍存在的异常者拖慢整体作业进度问题。提出了基于程序跟踪和聚类分析的异常者检测方法,设计了任务克隆、投机执行、细粒度任务分割相结合的执行加速策略。其中,程序跟踪使用了二进制代码插桩技术,并通过了类似采样的方式避免了对调用频繁的程序入口点的插桩将程序跟踪的运行时开销降到了实际可用的程度。基于异常点检测方法和执行加速策略的任务调度器在 ProActive 上的评测结果显示对于两类典型应用都取得了很好的加速效果,而用于加速任务的竞价型实例只使用了少量的计算成本。2)对于使用副本状态机进行容错的分布式服务,主要研究了在竞价型云计算平台中其可用性的变化与计算。针对竞价型实例的失效特点,分析了竞价型实例的市场价格与竞价的设定对竞价型实例失效概率的影响。将在竞价型云计算平台中部署分布式服务的可用性与计算成本优化问题形式化为一个非线性规划模型,并基于该模型设计了一个保证分布式服务同时大幅降低成本的竞价框架。该框架使用竞价型实例在分布式锁服务和分布式存储服务的案例中相比按需型实例节省了大量的计算成本,而分布式服务的可用性保持了同按需型实例相同的水平。3)对于需要高可用和高性能的在线服务,针对已有工作在可用性和性能上的诸多限制,提出了一个轻量级的基于双机暖备的解决方案。现有的基于嵌套虚拟化和虚拟机迁移等技术的方法本质上是一个冷备方案,受限于短暂的竞价实例回收预警时间,现有方案存在强制性的不可用缺陷。本文提出的方法利用了提供在线服务的节点互相备援,从而将已有方案服务迁移过程中新的按需实例的时间移出了关键路径。通过使用轻量级的进程活迁移,避免了嵌套虚拟化和定期内存检查点带来的大量运行时开销。利用时间可控的磁盘镜像方法实现了利用本地存储进行高 IOPS 数据读写同时保证了数据持久性。评测结果显示相比已有方案在线服务的可用性提升了近一个数量级,在两个基准测试集 TPC-W 和 YCSB 上的也有大幅性能提升。

基于本文的工作,云租户可以对竞价型云平台在不同层面、不同的应用场景下加以利用。我们相信,这些工作对推动竞价型实例更广泛的应用有实际意义。与此同时,本文的部分工作还可以继续优化或加以扩展。对于利用竞价型实例加速大规模计算密集型并行计算任务的研究,其应用场景还可推广到拥有大量空闲计算资源的本地集群中。另外,也可以考虑将按需计算实例全部替换为竞价型实例以减少计算成本,这引入了竞价型实例失效到大规模并行计算任务中。如何在节点可能大量失效的情况下保证作业完成时间是这个扩展需要解决的问题。一个可能的思路是调整加速执行的策略,针对作业中的每个任务都使用任务克隆机制在多个竞价型实例上同时执行,同时辅以投机执行等策略解决异常者问题。对于在竞价型云计算平台中针对使用副本状态机的分布式服务的竞价框架,可以考虑将其推广到在线服务的场景中。Cui 等人 \cite{Cui:2015:PAM:2815400.2815427} 已经提出了可以透明复制通用服务器程序执行的状态机复制系统,并在 Apache,MySQL 等常用服务器程序上进行了验证。如果使用 SMR 进行在线服务的容错是可行的,那么理论上将竞价框架扩展过来也是可行的。对于在线服务的轻量级容错机制,也可以考虑将其实施在分布式服务的节点上。每个节点的可用性提高后,分布式服务的可用性也得以保证。这些可行的扩展是本文未来工作中重点研究的方向。