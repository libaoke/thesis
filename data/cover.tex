\thusetup{
  %******************************
  % 注意:
  %   1. 配置里面不要出现空行
  %   2. 不需要的配置信息可以删除
  %******************************
  %
  %=====
  % 秘级
  %=====
  secretlevel={绝密},
  secretyear={2100},
  %
  %=========
  % 中文信息
  %=========
  ctitle={竞价型云平台系统可用性及性能提升技术研究},
  cdegree={工学博士},
  cdepartment={计算机科学与技术系},
  cmajor={计算机科学与技术},
  cauthor={郭维超},
  csupervisor={郑纬民教授},
  %cassosupervisor={陈文光教授}, % 副指导老师
  %ccosupervisor={某某某教授}, % 联合指导老师
  % 日期自动使用当前时间,若需指定按如下方式修改:
  % cdate={超新星纪元},
  %
  % 博士后专有部分
  cfirstdiscipline={计算机科学与技术},
  cseconddiscipline={系统结构},
  postdoctordate={2009年7月——2011年7月},
  id={编号}, % 可以留空: id={},
  udc={UDC}, % 可以留空
  catalognumber={分类号}, % 可以留空
  %
  %=========
  % 英文信息
  %=========
  etitle={Improving System Availability and Performance on Spot Clouds},
  % 这块比较复杂,需要分情况讨论:
  % 1. 学术型硕士
  %    edegree:必须为Master of Arts或Master of Science(注意大小写)
  %             “哲学、文学、历史学、法学、教育学、艺术学门类,公共管理学科
  %              填写Master of Arts,其它填写Master of Science”
  %    emajor:“获得一级学科授权的学科填写一级学科名称,其它填写二级学科名称”
  % 2. 专业型硕士
  %    edegree:“填写专业学位英文名称全称”
  %    emajor:“工程硕士填写工程领域,其它专业学位不填写此项”
  % 3. 学术型博士
  %    edegree:Doctor of Philosophy(注意大小写)
  %    emajor:“获得一级学科授权的学科填写一级学科名称,其它填写二级学科名称”
  % 4. 专业型博士
  %    edegree:“填写专业学位英文名称全称”
  %    emajor:不填写此项
  edegree={Doctor of Philosophy},
  emajor={Computer Science and Technology},
  eauthor={Weichao Guo},
  esupervisor={Professor Weimin Zheng},
  %eassosupervisor={Chen Wenguang},
  % 日期自动生成,若需指定按如下方式修改:
  % edate={December, 2005}
  %
  % 关键词用“英文逗号”分割
  ckeywords={竞价型云平台, 可用性, 性能, 成本效益},
  ekeywords={spot clouds, availability, performance, cost efficiency}
}

% 定义中英文摘要和关键字
\begin{cabstract}
  为了进一步提升计算资源利用率,Amazon EC2 云计算平台推出了竞价型虚拟机实例。竞
  价型实例不再是按需付费,其价格随市场供需关系变化,当用户设定的竞价低于市场价格
  时云平台将回收该实例。竞价型云平台的出现为云租户提供了利用低成本计算资源的机会,
  同时也引入了竞价型实例不可靠的特点。

  如何利用竞价型实例同时消除其带来的不可靠因素,这涉及到系统设计、容错机制和策略、
  竞价和可用区选择策略等多方面的协同。同时,不同的系统类型,如:批处理任务、大规
  模并行作业、Web 服务等,也有着不同的执行特点和对可用性的要求。

  从如何利用竞价型实例加速大规模计算密集型并行任务,到在竞价型云平台中提供分布式
  服务和在线服务,本文研究了如何提升竞价型云平台中的系统可用性及性能。主要创新点
  和研究成果包括:
  \begin{itemize}
    \item \emph{大规模计算密集型并行任务调度器。} 针对大规模计算密集型并行任务
    中存在的异常节点拖慢整体作业进度问题,提出了基于程序插桩跟踪和聚类分析的异常
    节点检测方法,设计了结合任务克隆、投机执行、细粒度任务分割等策略的执行加速机
    制。实验评测表明,该技术利用约占虚拟集群3\%计算成本的竞价型实例取得了平均30\%
    的作业完成时间缩减。
    \item \emph{竞价云平台中分布式服务的可用性与成本模型。} 指出了竞价型实例给
    基于SMR(状态机复制)的分布式服务的可用性分析带来的变化,通过竞价型实例的失效
    概率模型将竞价决策同分布式服务的可用性联系起来,将分布式服务的可用性与成本最
    优化抽象为一个非线性规划问题。并据此设计了可用性与成本兼顾的竞价框架,在保持
    可用性同按需型实例相同的情况下实现了超过80\%的成本节约。
    \item \emph{在线服务在竞价型云平台上的轻量级暖备方案。} 针对已有解决方案在
    可用性和性能上的限制,使用双机暖备、互备故障转移机制将不可控的虚拟机实例启动
    时间移出了服务迁移的关键路径,利用轻量级的进程迁移和时间可控的磁盘镜像技术提
    升了服务性能,综合考虑稳定性和成本的可用区选择方法减少了服务迁移次数。实验评
    测显示,相比已有方案在可用性有近一个数量级的提升,在TPC-W和YCSB基准测试集上
    的性能提升分别为45\%和100\%。
  \end{itemize}

\end{cabstract}

% 如果习惯关键字跟在摘要文字后面,可以用直接命令来设置,如下:
% \ckeywords{\TeX, \LaTeX, CJK, 模板, 论文}

\begin{eabstract}
   An abstract of a dissertation is a summary and extraction of research work
   and contributions. Included in an abstract should be description of research
   topic and research objective, brief introduction to methodology and research
   process, and summarization of conclusion and contributions of the
   research. An abstract should be characterized by independence and clarity and
   carry identical information with the dissertation. It should be such that the
   general idea and major contributions of the dissertation are conveyed without
   reading the dissertation.

   An abstract should be concise and to the point. It is a misunderstanding to
   make an abstract an outline of the dissertation and words ``the first
   chapter'', ``the second chapter'' and the like should be avoided in the
   abstract.

   Key words are terms used in a dissertation for indexing, reflecting core
   information of the dissertation. An abstract may contain a maximum of 5 key
   words, with semi-colons used in between to separate one another.
\end{eabstract}

% \ekeywords{\TeX, \LaTeX, CJK, template, thesis}
