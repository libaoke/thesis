\thusetup{
  %******************************
  % 注意:
  %   1. 配置里面不要出现空行
  %   2. 不需要的配置信息可以删除
  %******************************
  %
  %=====
  % 秘级
  %=====
  secretlevel={绝密},
  secretyear={2100},
  %
  %=========
  % 中文信息
  %=========
  ctitle={竞价云实例应用性能与可用性\\提升技术研究},
  cdegree={工学博士},
  cdepartment={计算机科学与技术系},
  cmajor={计算机科学与技术},
  cauthor={郭维超},
  csupervisor={郑纬民教授},
  %cassosupervisor={陈文光教授}, % 副指导老师
  %ccosupervisor={某某某教授}, % 联合指导老师
  % 日期自动使用当前时间,若需指定按如下方式修改:
  % cdate={超新星纪元},
  %
  % 博士后专有部分
  %cfirstdiscipline={计算机科学与技术},
  %cseconddiscipline={系统结构},
  %postdoctordate={2009年7月——2011年7月},
  %id={编号}, % 可以留空: id={},
  %udc={UDC}, % 可以留空
  %catalognumber={分类号}, % 可以留空
  %
  %=========
  % 英文信息
  %=========
  etitle={Improving Performance and Availability of Applications on Spot Clouds},
  % 这块比较复杂,需要分情况讨论:
  % 1. 学术型硕士
  %    edegree:必须为Master of Arts或Master of Science(注意大小写)
  %             “哲学、文学、历史学、法学、教育学、艺术学门类,公共管理学科
  %              填写Master of Arts,其它填写Master of Science”
  %    emajor:“获得一级学科授权的学科填写一级学科名称,其它填写二级学科名称”
  % 2. 专业型硕士
  %    edegree:“填写专业学位英文名称全称”
  %    emajor:“工程硕士填写工程领域,其它专业学位不填写此项”
  % 3. 学术型博士
  %    edegree:Doctor of Philosophy(注意大小写)
  %    emajor:“获得一级学科授权的学科填写一级学科名称,其它填写二级学科名称”
  % 4. 专业型博士
  %    edegree:“填写专业学位英文名称全称”
  %    emajor:不填写此项
  edegree={Doctor of Philosophy},
  emajor={Computer Science and Technology},
  eauthor={Weichao Guo},
  esupervisor={Professor Weimin Zheng},
  %eassosupervisor={Chen Wenguang},
  % 日期自动生成,若需指定按如下方式修改:
  % edate={December, 2005}
  %
  % 关键词用“英文逗号”分割
  ckeywords={竞价型云平台, 可用性, 性能, 成本效益},
  ekeywords={spot clouds, availability, performance, cost efficiency}
}

% 定义中英文摘要和关键字
\begin{cabstract}
  为了提升计算资源利用率,Amazon EC2 云计算平台推出了竞价型虚拟机实例。与按需
  付费方式不同,其价格随市场供需关系变化。当云租户设定的竞价低于市场价格时,Amazon 
  EC2 将回收该实例。竞价型实例的出现为云租户提供了利用低成本计算资源的机会,同
  时也引入了易失效的不可靠特点。

  如何在保留竞价型实例廉价特征的同时,消除其带来的不可靠因素?这涉及到对竞价
  型实例上应用的架构设计、容错机制、竞价策略等多方面的协同。不同的应用类型有
  着不同的执行特点和对可用性及性能的要求。从利用竞价型实例加速大规模计算密集型并行
  任务,到在竞价型云平台中提供分布式服务和在线服务,本文从云租户的角度研究了如
  何提升竞价云实例应用的性能与可用性。主要创新点和研究成果包括:
  \begin{itemize}
    \item \emph{大规模计算密集型并行任务利用竞价型实例的加速方法。} 针对大规
    模计算密集型并行任务中存在的异常节点拖慢整体作业进度问题,提出了基于程序插桩跟
    踪和聚类分析的异常节点检测方法,设计了结合任务克隆、投机执行、细粒度任务分割策
    略的执行加速机制。实验评测表明,该技术利用竞价型实例取得了平均30\%的作业完成时
    间缩减。
    \item \emph{竞价型实例上分布式服务的可用性与成本模型。} 指出了竞价型实例给基
    于状态机复制(SMR)的分布式服务的可用性分析带来的变化,通过竞价型实例的失效概率
    模型将竞价决策同分布式服务的可用性联系起来,并将分布式服务的可用性与成本的最优化
    抽象为一个非线性规划问题。据此设计了可用性与成本兼顾的使用竞价型实例提供分布式服
    务的竞价框架,在保持可用性同相应按需型实例相当的情况下实现了超过80\%的成本节约。
    \item \emph{竞价型云平台中在线服务的轻量级暖备机制。} 针对已有解决方案在
    可用性和性能上的限制,使用双机暖备、互备故障转移机制将不可控的虚拟机实例启动时
    间移出服务迁移的关键路径,利用轻量级的进程迁移和时间可控的磁盘镜像技术提升了服
    务性能。评测显示,相比已有方法在可用性上有近一个数量级的提升,在TPC-W和YCSB基
    准测试集上的性能提升分别达到45\%和100\%。
  \end{itemize}

\end{cabstract}

% 如果习惯关键字跟在摘要文字后面,可以用直接命令来设置,如下:
% \ckeywords{\TeX, \LaTeX, CJK, 模板, 论文}

\begin{eabstract}
   
   For increasing the computing resource utilization, Amazon EC2 proposed the 
   spot instances. Unlike their on-demand counterparts, the price of spot instances 
   varies in real-time based on supply and demand. The spot instances will be revoked 
   when its price exceeds the tenant's bid. Spot instances bring an opportunity 
   of greater cost efficiency for tenants, but the unreliability at the same time.

   Utilizing spot instances and masking its out-of-bid failures involve 
   system designs, fault tolerance mechanisms, and bidding strategies. Different 
   applications have different execution characteristic, and availability and 
   performance requirement. From speeding up massive compute-intensive parallel tasks 
   with spot instances to providing distributed services and online services on spot 
   clouds, this dissertation presents a thorough study of improving availability and 
   performance of applications on spot clouds from a could tenant's perspective. Main 
   contributions of this dissertation are summarized in following: 

  \begin{itemize}
    \item Addressing on the outlier problem of massive compute-intensive parallel tasks, 
    we propose a binary instrumentation tracing and clustering analysis based outlier 
    detection scheme, and using spot instances speeding up the job with task cloning, 
    speculative execution, and task split. Experimental results show about 30\% job 
    complementation time reduction in average.
    \item We point out that the availability analysis of SMR based distributed services 
    with spot instances is different, and formalize the optimization of availability and 
    cost of distributed services with a non-linear programming model. Accordingly, we 
    design a cost and availability aware bidding framework, reducing more than 80\% cost 
    while keeping the same availability level compared with using on-demand instances.
    \item We present a warm standby based mechanism for hosting online services using 
    spot instances without availability and performance limitations. It eliminates forced 
    unavailability by removing the uncontrolled instance allocation from service migration, 
    and thus improves the service availability near an order of magnitude. Light-weight 
    migration and staleness bounded async disk mirroring gain 45\% and 100\% performance 
    improvement in TPC-W and YCSB benchmarks respectively.
  \end{itemize}

\end{eabstract}

% \ekeywords{\TeX, \LaTeX, CJK, template, thesis}
